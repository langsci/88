\addchap{List of abbreviations}
\begin{refsection}

\section*{Grammatical categories in interlinear morphemic translations}

\begin{multicols}{2}
\setlength{\parindent}{0pt}\textsc{abl}  ablative
	
	\textsc{abs}  absolutive
	
	\textsc{acc}  accusative
	
	\textsc{adjvr}  adjectivizer
	
	\textsc{all}  allative
	
	\textsc{an}  animate
	
	\textsc{art}  article
	
	\textsc{asp}  aspect
	
	\textsc{at}  attributor
	
	\textsc{aux}  auxiliary
	
	\textsc{cl}  noun class
	
	\textsc{coll}  collective
	
	\textsc{compl}  completive
	
	\textsc{conn}  connective
	
	\textsc{cont}  continuative
	
	\textsc{cop}  copula
	
	\textsc{d}\oldstylenums{1}  determiner of first person deixis
	
	\textsc{d}\oldstylenums{3}  determiner of third person deixis
	
	\textsc{dat}  dative
	
	\textsc{def}  definite
	
	\textsc{dem}  demonstrative
	
	\textsc{des}  desiderative
	
	\textsc{det}  determiner
	
	\textsc{dir}  directional
	
	\textsc{du}  dual
	
	\textsc{dyn}  dynamic
	
	\textsc{el}  elative
	
	\textsc{erg}  ergative
	
	\textsc{exist}  existence
	
	\textsc{f}  feminine
	
	\textsc{fin}  finite
	
	\textsc{foc}  focus
	
	\textsc{fut}  future
	
	\textsc{gen}  genitive

%\end{multicols}
%\begin{multicols}{2}
	
	%\setcounter{page}{1}
	\textsc{ger}  gerund
	
	\textsc{hab}  habitual
	
	\textsc{hon}  honorific
	
	\textsc{hum}  human
	
	\textsc{ill}  illative
	
	\textsc{imp}  imperative
	
	\textsc{ind}  indefinite
	
	\textsc{indep}  independent
	
	\textsc{iness}  inessive
	
	\textsc{inf}  infinitive
	
	\textsc{inst}  instrumental
	
	\textsc{int}  interrogative
	
	\textsc{io}  indirect object
	
	\textsc{lat}  lative (${\simeq}$ directional)
	
	\textsc{loc}  locative
	
	\textsc{m}  masculine
	
	\textsc{mid}  middle voice
	
	\textsc{n}  neuter
	
	\textsc{neg}  negative
	
	\textsc{nhum}  non-human
	
	\textsc{nom}  nominative
	
	\textsc{nonsg}  non-singular
	
	\textsc{nr}  nominalizer
	
	\textsc{obj}  object (verb affix position)
	
	\textsc{obl}  oblique (affix position)
	
	\textsc{part}  participle
	
	\textsc{past}  past tense
	
	\textsc{perl}  perlative
	
	\textsc{pl}  plural
	
	\textsc{pol}  polite
	
	\textsc{poss}  possessor (nominal affix position)
	
	\textsc{praet}  praeterlative
	
	\textsc{pf}  perfect
	
	\textsc{prog}  progressive aspect
	
	\textsc{prs}  present
	
	\textsc{pst}  past
	
	\textsc{ptl}  particle
	
	\textsc{rdp}  reduplication
	
	\textsc{real}  realized
	
	\textsc{refl}  reflexive
	
	\textsc{rel}  relative
	
	\textsc{sbj}  subject (verb affix position)
	
	\textsc{sbjv}  subjunctive
	
	\textsc{sep}  separative
	
	\textsc{sg}  singular
	
	\textsc{sim}  simultaneous
	
	\textsc{sr}  subordinator
	
	\textsc{super}  superlative/-essive
	
	\textsc{term}  terminative
	
	\textsc{top}  topic
	
	\textsc{tr}  transitive
	
	\textsc{vol}  volitional
\end{multicols}

\section*{Abbreviations of language names}

\begin{multicols}{2}\setlength{\parindent}{0pt}
	
	
%	Engl.	English
%	
%	Germ./Ger.	German
%	
%	Hung.	Hungarian
	
	\textsc{ie} Indo-European
	
%	Ital./It.	Italian
%	
%	Jap.	Japanese
%	
%	Lat.	Latin
	
	\textsc{oe} Old English
	
	\textsc{ohg} Old High German
		
%	Port.	Portugese

	\textsc{pie} Proto-Indo-European 
	
%	Russ.	Russian
%
%	Span.	Spanish	
%	
%	Turk.	Turkish

	
%	Accadic
%	
%	Amharic
%	
%	Avestic
%	
%	Dyirbal
%	
%	English
%	
%	Finnish
%	
%	French
%	
%	German
%	
%	Hittite
%	
%	Hungarian
%	
%	Italian
%	
%	Japanese
%	
%	Kalkatungu
%	
%	Mandarin
%	
%	Mangarayi
%	
%	Nahuatl
%	
%	Portuguese
%	
%	Quechua
%	
%	Somali
%	
%	Swahili
%	
%	Tok Pisin
%	
%	Totonac
%	
%	Tswana
%	
%	Turkish
%	
%	Uzbek
%	
%	Vedic
	

\end{multicols}


\section*{Other abbreviations}
\begin{multicols}{2}\setlength{\parindent}{0pt}
	
		id. idem
	
		{op. cit.} opere citato
	
\end{multicols}

\printbibliography[heading=subbibliography]
\end{refsection}

\setlength{\parindent}{10pt}