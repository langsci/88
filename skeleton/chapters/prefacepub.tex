\addchap{Preface to the 1995/2002 editions}\label{95preface}

A preliminary version of the present work was distributed in 1982 under the following title: \textit{Thoughts on grammaticalization. A programmatic sketch. Vol. I.} Köln: Institut für Sprachwissenschaft der Universität (\textit{Arbeiten des Kölner Universalienprojekts}, 48). It got out of stock immediately, but has been in high demand since. A slightly revised version was released in January 1985, but only in form of a number of xerocopies. The original plan was, of course, to get back to work on grammaticalization as soon as possible, to write up volume \textsc{ii} and then publish the whole work. Then the title, too, would have been streamlined a bit. However, I never got around to do that.

The semipublished 1982 paper has played an instrumental role in the development of modern work on grammaticalization. Many people have asked me to at least make it available in published form, even if I should never manage to round it off. This is what I am doing here. Consequently, this publication is slightly anachronistic. I have removed those errors of the preliminary version that I got aware of. I have modified many points of detail. I have updated references to unpublished material. But I have not taken into consideration the vast amount of literature on grammaticalization that has appeared since (including my own more recent contributions) and that would lead me to reformulate substantially some of the ideas expounded here. Readers should be aware that the state of research reflected here is essentially that of 1982.

References to volume \textsc{ii}, including even the ``Prospect of contents of volume \textsc{ii}'', have not been deleted. A fair appreciation of what is being published here is only possible if one considers that it was always intended to be only half of what would, at least, be necessary. However, I doubt that volume \textsc{ii} will ever be published. Below, I list the articles on grammaticalization that I have published since 1982. Some of them may be considered to fill the lacunae created by the prospect. In particular, the following assignments may be allowed:

\begin{description}
\item[Ch. 5.2:]  \citet{Lehmann1989a}, \citet{Lehmann2002}.

\item[Ch. 6.3:]  \citet{Lehmann1989b}.

\item[Ch. 7.2:]  \citet{Lehmann1985b}, \citet{Lehmann1986}.

\item[Ch. 8:]    \citet{Lehmann1985a}, \citet{Lehmann1987}, \citet{Lehmann1992}.

\item[Ch. 9:]    \citet{Lehmann1993}, \citet{Lehmann1995}.
\end{description}


I have been unable to get my English grammar and style revised by a native speaker, and I must apologize for the inconveniences resulting therefrom. Finally, cordial thanks go to Cornelia Sünner for the effort she has made in editing the typescript. I also thank the numerous colleagues who have reacted to the preliminary version and whose comments would deserve fuller attention.

\vspace{\baselineskip}

\begin{minipage}{.45\linewidth}
	\begin{flushleft}
		\noindent Bielefeld, 21.07.1995
	\end{flushleft}
\end{minipage}
\begin{minipage}{.45\linewidth}
	\begin{flushright}
		Christian Lehmann
	\end{flushright}
\end{minipage}

\vspace{\baselineskip}

\noindent For the second edition, some changes and corrections have been made.

\vspace{\baselineskip}

\begin{minipage}{.45\linewidth}
	\begin{flushleft}
		\noindent Erfurt, 08.07.2002
	\end{flushleft}
\end{minipage}
\begin{minipage}{.45\linewidth}
	\begin{flushright}
		Christian Lehmann
	\end{flushright}
\end{minipage}

%\printbibliography[heading=subbibliography]