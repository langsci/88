% Ch 01
\chapter{The history of research in grammaticalization} \label{chap:1}

As far as I can see, it was Antoine Meillet (\citeyear{Meillet1912}) who coined the term ``grammaticalization'' and first applied it to the concept for which it is still used today. We will return to him in a moment. The concept itself, however, and the ideas behind it, are considerably older. The idea that grammatical formatives evolve from lexemes, that affixes come from free forms, was already expounded by the French philosopher Étienne Bonnot de Condillac. In his work \textit{Essai sur l'origine des connaissances humaines} (\citeyear{Condillac1746}), he explained the personal endings of the verb through agglutination of personal pronouns and maintained that verbal tense came from the coalescence of a temporal adverb with the stem. Again, John Horne Tooke, in his etymological work Έπεα πτερόεντα \textit{or the diversions of Purley} (vol. I: 1786, vol. \textsc{ii}: 1805), claimed that prepositions derive from nouns or verbs.\footnote{Information on Condillac and Horne Tooke from \citet[109f, 132--134]{Arens1969}, and \citet[119, 452f]{Stammerjohann1975}.} We shall see in \sectref{chap:1} that all such processes do, in fact, occur, though not necessarily in the specific cases which these authors had in mind.

Condillac and Horne Tooke were certainly only forerunners to the first evolutive typologists, notably August Wilhelm von Schlegel and Wilhelm von Humboldt. In his \textit{Observations sur la langue et la littérature provençales} (\citeyear{Schlegel1818}), Schlegel deals extensively with the renewal of Latin synthetic morphology by Romance analytic morphology. About the formation of the latter, he writes:

\begin{quote}
C'est une invention en quelque façon négative, que celle qui a produit les grammaires analytiques, et la méthode uniformément suivie à cet égard peut se réduire à un seul principe. On dépouille certains mots de leur énergie significative, on ne leur laisse qu'une valeur nominale, pour leur donner un cours plus général et les faire entrer dans la partie élémentaire de la langue. Ces mots deviennent une espèce de papier-monnaie destiné à faciliter la circulation. (\citeyear[28]{Schlegel1818})
\end{quote}

\noindent This is followed by a series of Latin-Romance examples of different kinds, including the development of articles, auxiliaries and indefinite pronouns, which have subsequently become the stock examples of grammaticalization theory. Although Schlegel goes so far as to speak of “la formation d'une nouvelle grammaire” (\citeyear[30]{Schlegel1818}), he views the development essentially as due to linguistic decadence. It will be observed, however, that some of the core aspects of grammaticalization, viz. semantic depletion and expansion of distribution, are foreshadowed here.

Wilhelm von Humboldt arrived at more far-reaching conclusions. In his academic lecture on the origins of grammatical forms, he proposed that “grammatische Bezeichnung” (the signifying of grammatical categories, as opposed to objects) evolves through the following four stages (\citeyear[54f]{Humboldt1822}):

%\setcounter{page}{1}
\begin{enumerate}
\item[\textsc{i}.] “grammatische Bezeichnung durch Redensarten, Phrasen, Sätze”: grammatical categories are completely hidden in the lexemes and in the semantosyntactic configurations.

\item[\textsc{ii}.] “grammatische Bezeichnung durch feste Wortstellungen und zwischen Sach- und Formbedeutung schwankende Wörter”.

\item[\textsc{iii}.] “grammatische Bezeichnung durch Analoga von Formen”: here the “vacillating words” have been agglutinated as affixes to the main words. The resulting complexes are not “forms”, unitary wholes, but only “aggregates”, and therefore mere “analogs to forms”.

\item[\textsc{iv}.] “grammatische Bezeichnung durch wahre Formen, durch Beugung und rein grammatische Wörter”.
\end{enumerate}


\noindent These four stages are connected with each other “durch verloren gehende Bedeutung der Elemente und Abschleifung der Laute in langem Gebrauch.”

One may simply overlook the “evaluation” of the different stages to which this theory is committed. One may also regard it as a terminological issue whether the term ‘grammatical form’ can be correctly applied only at stage \textsc{iv}, and not also at the other stages. But one must recognize that this account of the evolution of grammatical forms is essentially a theory of grammaticalization, if only a sketchy one. Three things are worth noting here. First, the term ‘grammatical form’ must not mislead one into thinking that this theory deals only with the  expression of the language sign. The passages quoted leave no doubt that the evolution in question affects both the meaning and the expression of the grammatical sign. Secondly, the four stages are essentially the morphological types of the linguistic typology of the time: stages I and/or \textsc{ii} = isolating, \textsc{iii} = agglutinative, \textsc{iv} = flexional. Thirdly, linguistic typology, which in the twentieth century was reduced to a synchronic discipline, is here conceived as evolutive typology. Consequently, the theory of grammaticalization is tied, from the very start, to evolutive typology.

This theory was subsequently widely received under the name of “Agglutinationstheorie”.\label{Agglutinationstheorie} This term appears to refer only to the transition towards stage \textsc{iii}, but was later used to comprise all of the four stages.\footnote{This inadequacy of the term was also felt by Jespersen, who proposed to substitute it by “coalescence theory” (1922:376).} The first to apply the theory, Franz Bopp\label{Bopp}, who shared ideas with Humboldt through correspondence, actually concentrated on stage \textsc{iii}. In his \textit{Über das Conjugationssystem der Sanskritsprache in Vergleichung mit jenem der griechischen, lateinischen, persischen und germanischen Sprache} (\citeyear[147f]{Bopp1816}; apud \citet[177]{Arens1969}, and again in vol.~\textsc{i} of his \textit{Vergleichende Grammatik des Sanskrit, Zend, Griechischen, Lateinischen, Litauischen, Altslavischen, Gothischen und Deutschen} \citep{Bopp1833}, he derived the personal endings of the Indo-European verb from agglutinated personal pronouns.\footnote{This application of agglutination theory is not to be confused with Bopp's typology of roots.} Several of the neogrammarians, among them Brugmann, were favorably inclined to hypotheses of this kind. Again, the typological version of agglutination theory was most vigorously promoted by August Schleicher; he followed Humboldt in making agglutination theory the center of his evolutive typology.

Another prominent representative of agglutination theory is Georg von der Gabelentz. The essential passage from his \textit{Die Sprachwissenschaft}, which remained unaltered in the second edition (\citeyear[251]{Gabelentz1891} = 1901:256), will be quoted in full here, because it summarizes well what was known or thought about agglutination theory at that time.

\begin{quote}
	Nun bewegt sich die Geschichte der Sprachen in der Diagonale zweier Kräfte: des Bequemlichkeitstriebes, der zur Abnutzung der Laute führt, und des Deutlichkeitstriebes, der jene Abnutzung nicht zur Zerstörung der Sprache ausarten läßt. Die Affixe verschleifen sich, verschwinden am Ende spurlos; ihre Funktionen aber oder ähnliche drängen wieder nach Ausdruck. Diesen Ausdruck erhalten sie, nach der Methode der isolierenden Sprachen, durch Wortstellung oder verdeutlichende Wörter. Letztere unterliegen wiederum mit der Zeit dem Agglutinationsprozesse, dem Verschliffe und Schwunde, und derweile bereitet sich für das Verderbende neuer Ersatz vor: periphrastische Ausdrücke werden bevorzugt; mögen sie syntaktische Gefüge oder wahre Komposita sein (englisch: \textit{I shall see}, — lateinisch \textit{videbo = vide-fuo}); immer gilt das Gleiche: die Entwicklungslinie krümmt sich zurück nach der Seite der Isolation, nicht in die alte Bahn, sondern in eine annähernd parallele. Darum vergleiche ich sie der Spirale.
\end{quote}
The extent to which Gabelentz is obliged to Humboldt emerges clearly from this quotation. On the other hand, two things are new here: First, an explanation for grammaticalization is offered, this being seen as the result of two competing forces, the tendency towards ease of articulation and the tendency towards distinctness. We will meet these again and again, in various disguises, in the subsequent literature. Secondly, the evolution is not conceived as linear, as leading from a primitive to an advanced stage, but as basically cyclic, though Gabelentz is cautious enough to use the more precise metaphor of the spiral. With the necessary refinements, this still corresponds to the most recent insights.

In \citeyear{Meillet1912}, Antoine Meillet published his article “L'évolution des formes grammaticales”. Although the title is reminiscent of Humboldt's lecture, Meillet shows no sign of being acquainted with it or with agglutination theory, though he certainly must have been. In particular, his examples include Schlegel's examples. However, grammaticalization was of interest to him not for its typological implications, but for its capacity to explain certain facts in the history of Indo-European languages. He thus continues the Bopp--neogrammarian tradition. Meillet assumes three main classes of words, “mots principaux”, “mots accessoires” and “mots grammaticaux”, between which there is a gradual transition.

\begin{quote}
L'affaiblissement du sens et l'affaiblissement de la forme des mots accessoires vont de pair; quand l'un et l'autre sont assez avancés, le mot accessoire peut finir par ne plus être qu'un élément privé de sens propre, joint à un mot principal pour en marquer le rôle grammatical. Le changement d'un mot en élément grammatical est accompli. (\citeyear[139]{Meillet1912}). \label{meillet}
\end{quote}

\noindent This leads Meillet to what appears to be a reformulation of Gabelentz's agglutination theory:

\begin{quote}
Les langues suivent ainsi une sorte de développement en spirale; elles ajoutent de mots accessoires pour obtenir une expression intense; ces mots s'affaiblissent, se dégradent et tombent au niveau de simples /141/ outils grammaticaux; on ajoute de nouveaux mots ou des mots différents en vue de l'expression; l'affaiblissement recommence, et ainsi sans fin.
\end{quote}

\noindent The two driving factors he mentions, “expressivité” and “usage”, also have much in common with Gabelentz's tendency towards distinctness and towards ease. Even when he contends that analytic (= periphrastic) and synthetic constructions do not differ in principle, because they are connected through grammaticalization, citing the example of the Latin-Romance tenses, he only seems to strengthen a point that was already implicit in agglutination theory. However, Meillet does go beyond this. First, he introduces the term ``grammaticalization'' (\citeyear[133]{Meillet1912}), though he consistently puts it in quotation marks. He does not define the term, but uses it in the sense of “attribution du caractère grammatical à un mot jadis autonome” (\citeyear[131]{Meillet1912}). Secondly, Meillet opposes grammaticalization to analogy as the two principal processes of grammatical change (s. below § 5.4), thus assigning grammaticalization a more narrowly defined place in linguistic theory. And finally (\citeyear[147f]{Meillet1912}), he offers what appears to be a useful extension of this notion: he considers that the order of constituents may be grammaticalized, too, illustrating from Latin, in which word order signifies expressive nuances, and French, where it expresses syntactic relations.

Three years later, in his article “Le renouvellement des conjonctions”, Meillet extends his theory to the historical analysis of conjunctions, especially in Latin-Romance. The recruitment of new words which are then to follow the paths of grammaticalization already well established in the language, is termed “renouvellement” and distinguished from “création”, where grammatical and/or formal categories previously absent from the language are introduced. The substitution of Latin \textit{nam} by \textit{quare} {\textgreater} French \textit{car} is an example of ``renovation'' (renewal).\footnote{``Renovation'' will here be used instead of the traditional ``renewal'' because it offers a neat counterpart to ``innovation''.}

Continuing in chronological order, we next come to Edward Sapir, who again represents the other, Humboldtian tradition. Sapir's primary interest was neither in grammaticalization as a force in historical change (he does not use the term) nor in agglutination theory or evolutive typology; but in establishing a continuum of the different kinds of linguistic concepts as a basis for his synchronic typology, he actually contributes to both of these theories. In ch.~V of his \textit{Language}, \citet[102]{Sapir1921} defines the following four classes of concepts:\label{SapirHumCh4}

\begin{table}[H] % I have inserted H because the text still needs the table to be at exactly this location to make sense
\begin{tabular}{llll}
\lsptoprule
{Material content} &  & \textsc{i}. & Basic Concepts\\
 &  & \textsc{ii}. & Derivational Concepts\\
\midrule
{Relational} &  & \textsc{iii}. & Concrete Relational Concepts\\
&  & \textsc{iv}. & Pure Relational Concepts.\\
\lspbottomrule
\end{tabular}
%\caption{Insert caption here}
\end{table}

\newpage
\noindent Semantically, there is a gradience through these four classes from the concrete to the abstract; morphologically, there is a parallel gradience from “independent words or radical elements” to expression “by affixing non-radical elements to radical elements ... or by their inner modification, by independent words, or by position”. Sapir also mentions the possibility of a word's diachronic passage through this continuum. His most important, and most problematic, innovation is his attempt to give a more precise semantic basis to the different grammaticalization stages. In this, he has had practically no followers.\label{page6} One point which might at first seem to be of minor importance is noteworthy: the expression of grammatical concepts by “position” shows up at the end of Sapir's scale, while it appeared at the beginning of Humboldt's four stages. Take this together with Meillet's contention that word order may be grammaticalized, too, and the problem becomes obvious.

Henri Frei's work may be mentioned in passing. Nothing in his book \textit{La grammaire des fautes} (\citeyear{Frei1929}) is intended to be a contribution to grammaticalization theory; but he does adduce a lot of relevant data for “un passage incessant du signe expressif au signe arbitraire”, for which he finds two forces responsible, “le besoin d'expressivité” and “la loi de l'usure” (\citeyear[233]{Frei1929}). Frei's association of grammaticalization with a change from the expressive to the arbitrary will yet occupy us (\citeyear[115]{Frei1929}).

In the period of American and even of European structuralism, topics such as grammaticalization were not fashionable. With the decline of morphological and evolutive typology, this vein of research in grammaticalization virtually broke off. The only work of this time in which agglutination theory figures prominently is the Africanist Carl Meinhof's book \textit{Die Entstehung flektierender Sprachen} (\citeyear{Meinhof1936}), in which he treats the evolution of flexional morphology in Semitic, Hamitic and Indo-European languages. Following Jespersen (\citeyear[375--388]{Jespersen1922}; see ch.~V.4), Meinhof in § 4 posits two principal ways in which inflection can evolve: (1) through grammaticalization, for instance of nouns or verbs via postpositions to case suffixes; or (2) through the reinterpretation of already existing phonological outgrowths of the word.

Apart from this sporadic recurrence, however, agglutination theory does not, as far as I can see, regain its former popularity until \citet{Hodge1970} and \citet{Givón1971} (the latter apparently being unaware of the venerable tradition which he continues). Two important articles which throw new light on grammaticalization are Roman Jakobson's “Boas's view of grammatical meaning” (\citeyear{Jakobson1959}) and V. M. Žirmunskij's “The word and its boundaries” (\citeyear{Žirmunskij1966}; Russian original 1961). Jakobson attributes to Boas a distinction between “those concepts which are grammaticalized and consequently obligatory in some languages but lexicalized and merely optional in others” (\citeyear[492]{Jakobson1959}), adducing “the obligatoriness of grammatical categories as the specific feature which distinguishes them from lexical meanings” \label{quote:Jakobson} (\citeyear[489]{Jakobson1959}). This is clearly an advance because it adds an essential syntactic aspect to the until then almost exclusively morphological view of grammaticalization. Here for the first time, too, an opposition between grammaticalization and lexicalization is formulated.

In § 3 of his article, Žirmunskij deals with the “unification of the word combination into a single (compound) word.” There are two possible directions that this process can take: either towards grammaticalization, which yields “a specific new analytical form of the word”, or towards lexicalization, which yields “a phraseological equivalent of the word in the semantic sense.” (\citeyear[83]{Žirmunskij1966}) In the first case, the next stage is a synthetic inflectional word form; in the second case, the next stage is a compound word. Several points should be stressed here. First, there are processes of unification which do not involve the development of one element of the combination into a grammatical formative and which are therefore not regarded as grammaticalization. Second, such processes are called lexicalization. Observe that this use of the term “lexicalization” is quite different from Jakobson's use quoted above; this will constitute one of our problems (§~5.2). Thirdly, the term ``grammaticalization'' is used here not (only) for the transition from the analytic to the synthetic construction, i.e. the agglutination process, but is explicitly applied to the formation of an analytic construction. This is consistent with the meaning of the term which covers an open-ended continuum comprising all of Humboldt's or Sapir's four stages.

Outside structuralism, the Indo-Europeanist tradition of grammaticalization theory remained uninterrupted. Its most important representatives are Jerzy Kuryłowicz and Emile Benveniste. Kuryłowicz applied the concept of grammaticalization systematically in his book \textit{The inflectional categories of Indo-European}, many of which are explained through grammaticalization. In his article “The evolution of grammatical categories” (\citeyear{Kuryłowicz1965}; notice again the tradition of article titles!), Kuryłowicz defines:

\begin{quote} \label{quote:kurylowciz}
Grammaticalization consists in the increase of the range of a morpheme advancing from a lexical to a grammatical or from a less grammatical to a more grammatical status, e.g. from a derivative formant to an inflectional one. (\citeyear[52]{Kuryłowicz1965})
\end{quote}

\noindent By ‘increasing range’ Kuryłowicz means wider distribution, a defining factor of grammaticalization which had hitherto only been hinted at by Schlegel. Notice that word-formation is reintroduced into the picture, which we might think to have excluded from grammaticalization with Žirmunskij. Kuryłowicz then gives a survey of various Indo-European grammatical categories and their development through grammaticalization. He also opposes grammaticalization to lexicalization in a third sense which will occupy us in §~5.2.

Benveniste, who, curiously enough, consistently avoids the term ``grammaticalization'', has made various contributions to the subject. In his article “Mutations of linguistic categories” (\citeyear{Benveniste1968}), he takes up Meillet's distinction between “création” and “renouvellement”, explaining that the former is innovative change, where grammatical categories may disappear or emerge for the first time, while the latter is conservative change, where categories are only formally ``renovated''. The examples are again the same as in \citet{Meillet1912}: the Latin-Romance perfect and future.

Switching back, for the last time, to the conception of evolutive typology, we find this revived in two articles by Carleton T. Hodge and Talmy Givón. In his paper “The linguistic cycle” (\citeyear{Hodge1970}), Hodge somewhat simplifies the picture by distinguishing only two stages, one with heavy syntax and little morphology (Sm), which roughly comprises Humboldt's stages I and \textsc{ii}; and another with little syntax and heavy morphology (sM), which corresponds to Humboldt's stages \textsc{iii} and \textsc{iv}. His point is essentially an empirical one: he adduces the history of Egyptian as factual proof for the hypothesis that a single language can pass through a full cycle ‘sM → Sm → sM’. His slogan “that one man's morphology was an earlier man's syntax” (3) is echoed in Givón's formulation “Today's morphology is yesterday's syntax.” (\citeyear[413]{Givón1971}), which is the central thesis of his article “Historical syntax and synchronic morphology: An archaeologist's field trip”. We will deal in §~8.3 with the role of grammaticalization in historical reconstruction. Here it suffices to mention that Givón has expanded his theory in various works, proposing, in 1979, the grammaticalization scale which we will discuss in \sectref{sec:2.2}. The notion of grammaticalization has by now become widely known and is receiving ever greater interest. I will end my review here and discuss more recent work in thematically more specific connections.

Summing up, we can say that the theory of grammaticalization has been developed by two largely independent linguistic traditions, that of Indo-European historical linguistics and that of language typology. The moment has come, I think, where the two threads should be united. One tradition is conspicuously absent from this picture, namely that of structural linguistics, from de Saussure to our day. This is by no means an accident: whereas historical linguistics and typology have been concerned, from their beginning, with processes and continuous phenomena and thus could easily accommodate grammaticalization as a process which creates such phenomena, structural linguistics has tended to favour a static view of language and clear-cut binary distinctions. In §~6 we will try and see whether the perspective of grammaticalization cannot, in fact shed some light on problems traditional in structural linguistics.
