\section {References}
Aguado, Miquel \& Lehmann, Christian 1989, “Zur Grammatikalisierung der Klitika im Katalanischen.” Raible, Wolfgang (ed.) 1989, \textit{Romanistik, Sprachtypologie und Universalienforschung. Beiträge zum Freiburger Romanistentag 1987.} Tübingen: G. Narr (\textsc{tbl}, 332); 151-162.

Lehmann, Christian 1985, “Grammaticalization: Synchronic variation and diachronic change.” \textit{Lingua e Stile} 20:303-318.

Lehmann, Christian 1985, “The role of grammaticalization in linguistic typology.” Seiler, Hansjakob \& Brettschneider, Gunter (eds.), \textit{Language invariants and mental operations. International interdisciplinary conference held at Gummersbach/Cologne, Germany, Sept. 18-23, 1983.} Tübingen: G. Narr (\textsc{lus}, 5); 41-52.

Lehmann, Christian 1986, “Grammaticalization and linguistic typology.” \textit{General Linguistics} 26:3-23.

Lehmann, Christian 1987, “Sprachwandel und Typologie.” Boretzky, Norbert et al. (eds.), \textit{Beiträge zum 3. Essener Kolloquium über Sprachwandel und seine bestimmenden Faktoren, vom 30.9. - 2.10.1987 [sic; i.e. 1986] an der Universität Essen.} Bochum: N. Brockmeyer (Bochum-Essener Beiträge zur Sprachwandelforschung, 4); 201-225.

Lehmann, Christian 1989, “Grammatikalisierung und Lexikalisierung.” \textit{Zeitschrift für Phonetik, Sprachwissenschaft und Kommunikationsforschung} 42:11-19.

Lehmann, Christian 1989, “Markedness and grammaticalization.” Tomić, Olga M. (ed.), \textit{Markedness in synchrony and diachrony.} Berlin \& New York: Mouton de Gruyter (Trends in Linguistics, Studies and Monographs, 39); 175-190.

Lehmann, Christian 1991, “Grammaticalization and related changes in contemporary German.” Traugott, Elizabeth C. \& Heine, Bernd (eds.), \textit{Approaches to grammaticalization. 2 vols.} Amsterdam \& Philadelphia: J. Benjamins (Typological Studies in Language, 19); \textsc{ii}:493-535.

Lehmann, Christian 1992, “Word order change by grammaticalization.” Gerritsen, Marinel \& Stein, Dieter (eds.) 1992, \textit{Internal and external factors in syntactic change.} Berlin \& New York: Mouton de Gruyter (Trends in Linguistics, 61); 395-416.

Lehmann, Christian 1993, “Theoretical implications of processes of grammaticalization.” Foley, William A. (ed.) 1993, \textit{The role of theory in language description.} Berlin \& New York: Mouton de Gruyter (Trends in Linguistics, 69); 315-340.

Lehmann, Christian 1995, “Synsemantika.” Jacobs, Joachim et al. (eds.), \textit{Syntax}. Ein internationales Handbuch zeitgenössischer Forschung. Berlin: W. de Gruyter (Handbücher der Sprach- und Kommunikationswissenschaft, 9/2); \textsc{ii}:1251-1266.

Lehmann, Christian 2002, {\textquotedbl}New reflections on grammaticalization and lexicalization.{\textquotedbl} Wischer, Ilse \& Diewald, Gabriele (eds.), \textit{New reflections on grammaticalization.} Amsterdam \& Philadelphia: J. Benjamins (\textsc{tsl}, 49); 1-18.