\addchap{Preface to the 1982 working paper version}

As we will be going a long way, through involved and ramified discussions, until we arrive at something like a definition of grammaticalization, the reader who wants to know beforehand what this book is all about is asked to accept this as a preliminary characterization: Grammaticalization is a process leading from lexemes to grammatical formatives. A number of semantic, syntactic and phonological processes interact in the grammaticalization of morphemes and of whole constructions. A sign is grammaticalized to the extent that it is devoid of concrete lexical meaning and takes part in obligatory grammatical rules. A simple example is the development of the Latin preposition \textit{ad} ‘at, towards’ into the Spanish direct object marker \textit{a}.

It must be made clear at the outset that this treatment is preliminary, incomplete and imperfect. It presents little more than what has been found out in the two centuries in which the subject has been studied, and probably it contains even less than that, because I have been unable to take notice of all the relevant literature. I must also warn the reader that I have great conceptual difficulties with the present subject, and I will leave many questions open. The problem is not so much an empirical one: there are sufficient analyzed data, and the empirical phenomena in themselves appear to be reasonably clear. What is highly unclear is how the phenomena are to be interpreted, classified and related to each other. Grammaticalization is such a pervasive process and therefore such a comprehensive notion that it is often difficult to say what does not fall under it. The present essay will therefore be concerned, first and foremost, with the question: what \textit{is} grammaticalization?

The discussion will not be couched in terms of a specific theory of grammar, one reason being that existing grammatical models are inadequate for the representation of the gradual nature which is essential to the phenomena comprised by grammaticalization. As many of the problems involved are traditional ones, they can be discussed in traditional terms.

The theory of language which is to account for the systematicity, goal-\linebreak directedness and dynamism inherent to grammaticalization must be structural, functional and operational in nature. It is essentially the theory of Wilhelm \citet{Humboldt1836}, which has been elaborated in more recent times by Eugenio \citet{Coseriu1974} and Hansjakob \citet{Seiler1978}. This theory has never been made fully explicit; but it will become transparent through all of the present treatment, and an attempt to make it more explicit will be presented in the last chapter.

The work is organized as follows. We start, in \chapref{chap:1}, with a brief historical review of the relevant literature. \chapref{chap:2} will supply some first clarifications to the concept of grammaticalization and will delimit it against related concepts. \chapref{chap:3} contains the bulk of the empirical data which illustrate grammaticalization, ordered according to semantically defined domains of grammar. From this evidence, the various basic processes which integrate grammaticalization and which are called its parameters are then extracted and ordered according to how they pertain to the paradigmatic or the syntagmatic aspect, to the content or the expression of the grammaticalized sign. The degree to which these parameters correlate will also be discussed in \chapref{chap:4}. The next chapter looks out for analogs to grammaticalization in different parts of the language system and tries to distinguish these from grammaticalization proper. In Chapter 6 we turn to a couple of traditional linguistic problems, asking whether the concept of grammaticalization can contribute anything towards their clarification. The various modes of contrasting different languages, including language typology and universals research, are discussed in the perspective of grammaticalization in Chapter 7. Chapter 8 concentrates on the diachronic aspect of grammaticalization, its role in language change and historical reconstruction. The final chapter tries to formulate the advances that may be made in language theory if grammaticalization is given its proper place in it.

Due to idiosyncrasies in the timing of my research projects, I have had to interrupt the writing of this book after \chapref{chap:4}. It was decided that the finished chapters should appear as volume I, while Chapters 5 -- 9 should be reserved for a second volume. I have included them in the preceding sketch and also given a prospect on their contents in order that the reader may get an idea of the plan of the complete work. It is my intention to complete volume II for \textit{akup} in 1983.

A cordial word of thanks goes to Bernd Heine and Mechthild Reh, who have been working on grammaticalization and evolutive typology, especially in African languages, simultaneously and partly in cooperation with me. They have been kind and disinterested enough to put their notes and manuscripts at my disposal. References are to this prepublication version; their work is now being published as \textit{akup} 47. Finally, I should like to thank Sonja Schlögel and Ingrid Hoyer, who have taken great care in typing and editing the manuscript.\enlargethispage{2\baselineskip}

\vspace{\baselineskip}

\begin{minipage}{.5\linewidth}
\begin{flushleft}
Cologne, 7.10.1982
\end{flushleft}
\end{minipage}
\begin{minipage}{.45\linewidth}
\begin{flushright}
Christian Lehmann
\end{flushright}
\end{minipage}