\addchap{Series Editors’ preface to the 3rd edition}

This is the third edition of Christian Lehmann’s \textit{Thoughts on grammaticalization}, a book which has had great impact on the development of grammaticalization studies since the 1980s in spite of its unusual publication history.

The first version was circulated in 1982 as a working paper of the University of Cologne’s UNITYP project. At that time, its full title included the indication ``Volume I'', because the author had planned a second volume. The first published edition appeared in 1995 with Lincom Europa. The second edition appeared again as a working paper, this time of the University of Erfurt (ASSidUE), in 2002.

We are happy that Christian Lehmann accepted our proposal to publish it again as a regular book. We feel that it deserves more prominence, as it provides an excellent overview of grammaticalization processes and its theoretical ideas have not been superseded.

The third edition contains few changes compared to the second edition from 2002; the main addition is an epilogue at the end of the book. Otherwise the author and editors limited themselves to a few stylistic modifications such as corrected typos, more consistent use of abbreviations and a few adaptations to the usual Language Science Press style.

Note that the post-1982 work that corresponds to the various chapters of the planned Volume II is mentioned in the preface of the 1995 edition.

\vspace{\baselineskip}
\begin{minipage}{.45\linewidth}
	\begin{flushleft}
		\noindent Berlin/Leipzig, October 2015  
	\end{flushleft}
\end{minipage}
\begin{minipage}{.45\linewidth}
	\begin{flushright}
		The Series Editors
	\end{flushright}
\end{minipage}                